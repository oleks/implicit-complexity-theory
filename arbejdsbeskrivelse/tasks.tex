% vim: set spell:

\section{Tasks}

The thesis consists of two parts: and introductory part, and a body of work.

The introductory part is expected to be completed within the first two working
months, and includes the following tasks:

\begin{enumerate}[(1)]

\item Survey previous work on implicit complexity theory and contrast it to the
explicit variant.

\item Choose a programming language, where all computation is bounded to P as a
baseline.

\item Choose a range of computational complexity classes under P, which could
be interesting in practice.

\item Extend the language with syntactic sugar, if useful to programming, and
constructs for annotating program sections with the complexity bounds chosen
above.\label{sugar}

\end{enumerate}

The rest of the thesis is concerned with implicitly categorising the chosen
complexity classes, and extending the language above, providing for effectful
bounded program sections, arriving perhaps, at a language useful in practice.
This includes the following tasks:

\begin{enumerate}[(1)]

\setcounter{enumi}{4}

\item Categorise the above computational complexity classes in an implicit
manner.

\item Prove the implicit categorisations equivalent to their explicit
counterparts.

\item Extend the language above, providing for bounded and effectful program
sections, where a programmer is limited to a subset of the language, reducible
to an implicit categorisation of the stated bound.\label{effectful}

\end{enumerate}

As we have noted above, the bounds are guaranteed trivially if we only allow
for a no-op in a bounded section. Task (\ref{effectful}) may therefore induce
further work on task (\ref{sugar}), and vice versa.
