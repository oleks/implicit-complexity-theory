% vim: set spell

\section{Tasks}

The thesis consists of two parts: and introductory part, and a body of work.

The introductory part is expected to be completed within the first working
month, and includes the following tasks:

\begin{enumerate}[(1)]

\item Survey previous work on implicit complexity theory and contrast it to the
explicit variant.

\item Choose and define a range of computational complexity classes, using the
conventional, explicit approach.

\item Devise a programming language, where the programmer can state, and be
statically guaranteed, a computational complexity bound on a program section,
e.g. the entire program.

\end{enumerate}

The rest of the thesis is concerned with implicitly categorising the chosen
complexity classes, and extending the language above, providing for effectful
bounded program sections, arriving perhaps, at a language useful in practice.
This includes the following tasks:

\begin{enumerate}[(1)]

\setcounter{enumi}{3}

\item Categorise the above computational complexity classes in an implicit
manner.

\item Prove the implcit categorisations equivalent to their explicit
counterparts.

\item Extend the language above, providing for bounded and effectful program
sections, where a programmer is limited to a subset of the language, reducible
to an implicit categorisation of the stated bound.

\end{enumerate}

The last task is expected to first include providing for effectful program
sections bounded to P. We then provide for more fine-grained, subpolynomial
bounds.
