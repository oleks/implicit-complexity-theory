% vim: set spell:

\section{Background}

% ``Implicit'' complexity theory is concerned with the design of programming
% languages for complexity-bounded computation. This is in contrast to
% ``explicit'' complexity theory, where we attempt to bound the computational
% complexity of already written programs.

 ``Implicit'' complexity theory is concerned with the design of programming
languages, such that programs are guaranteed by construction to be bounded in
their computational complexity. This is in contrast to ``explicit'' complexity
theory, where we attempt to bound the computational complexity of already
written programs.

Research in computational complexity has, so far, been predominantly focused on
the explicit variant. The hope with the implicit variant is
twofold\cite{baillot-et-al-2006}:

\begin{enumerate}[(a)]

\item provide for an alternative categorization of complexity classes,
expressed in terms of logical principles, rather than e.g. the costs induced
for a particular machine model; and

\item provide for a tractable approach to static verification of program
complexity.

\end{enumerate}

In theory, we are concerned with the influence of various logical principles
(e.g. elements of semantics and types) on the computational complexity of
programs. In practice, we are concerned with (de)restricting the programmer,
implying complexity guarantees.
